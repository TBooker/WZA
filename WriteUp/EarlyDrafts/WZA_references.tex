@article{Wakeley2003,
author = {Wakeley, John. and Lessard, S.},
journal = {Genetics},
number = {3},
pages = {1043--53},
title = {{Theory of the effects of population structure and sampling on patterns of linkage disequilibrium applied to genomic data from humans}},
volume = {164},
year = {2003}
}
@article{Stapley2017,
abstract = {Recombination, the exchange of DNA between maternal and paternal chromosomes during meiosis, is an essential feature of sexual reproduction in nearly all multicellular organisms. While the role of recombination in the evolution of sex has received theoretical and empirical attention, less is known about how recombination rate itself evolves and what influence this has on evolutionary processes within sexually reproducing organisms. Here, we explore the patterns of, and processes governing recombination in eukaryotes. We summarize patterns of variation, integrating current knowledge with an analysis of linkage map data in 353 organisms. We then discuss proximate and ultimate processes governing recombination rate variation and consider how these influence evolutionary processes. Genome-wide recombination rates (cM/Mb) can vary more than tenfold across eukaryotes, and there is large variation in the distribution of recombination events across closely related taxa, populations and individuals. We discuss how variation in rate and distribution relates to genome architecture, genetic and epigenetic mechanisms, sex, environmental perturbations and variable selective pressures. There has been great progress in determining the molecular mechanisms governing recombination, and with the continued development of newmodelling and empirical approaches, there is now also great opportunity to further our understanding of how and why recombination rate varies.},
author = {Stapley, Jessica and Feulner, Philine G.D. and Johnston, Susan E. and Santure, Anna W. and Smadja, Carole M.},
doi = {10.1098/rstb.2016.0455},
file = {:Users/s0784966/Library/Application Support/Mendeley Desktop/Downloaded/Stapley et al. - 2017 - Variation in recombination frequency and distribution across eukaryotes Patterns and processes.pdf:pdf},
issn = {14712970},
journal = {Philosophical Transactions of the Royal Society B: Biological Sciences},
keywords = {Adaptation,Crossing over,Evolution,Genetic linkage,Genomic architecture,Meiosis},
month = {dec},
number = {1736},
pmid = {29109219},
publisher = {Royal Society Publishing},
title = {{Variation in recombination frequency and distribution across eukaryotes: Patterns and processes}},
url = {https://royalsocietypublishing.org/doi/abs/10.1098/rstb.2016.0455},
volume = {372},
year = {2017}
}
@article{Pavy2012,
abstract = {In plants, knowledge about linkage disequilibrium (LD) is relevant for the design of efficient single-nucleotide polymorphism arrays in relation to their use in population and association genomics studies. Previous studies of conifer genes have shown LD to decay rapidly within gene limits, but exceptions have been reported. To evaluate the extent of heterogeneity of LD among conifer genes and its potential causes, we examined LD in 105 genes of white spruce (Picea glauca) by sequencing a panel of 48 haploid megagametophytes from natural populations and further compared it with LD in other conifer species. The average pairwise r 2 value was 0.19 (s.d.=0.19), and LD dropped quickly with a half-decay being reached at a distance of 65 nucleotides between sites. However, LD was significantly heterogeneous among genes. A first group of 29 genes had stronger LD (mean r 2 = 0.28), and a second group of 38 genes had weaker LD (mean r 2=0.12). While a strong relationship was found with the recombination rate, there was no obvious relationship between LD and functional classification. The level of nucleotide diversity, which was highly heterogeneous across genes, was also not significantly correlated with LD. A search for selection signatures highlighted significant deviations from the standard neutral model, which could be mostly attributed to recent demographic changes. Little evidence was seen for hitchhiking and clear relationships with LD. When compared among conifer species, on average, levels of LD were similar in genes from white spruce, Norway spruce and Scots pine, whereas loblolly pine and Douglas fir genes exhibited a significantly higher LD. {\textcopyright} 2012 Macmillan Publishers Limited All rights reserved.},
author = {Pavy, N. and Namroud, M. C. and Gagnon, F. and Isabel, N. and Bousquet, J.},
doi = {10.1038/hdy.2011.72},
file = {:Users/s0784966/Library/Application Support/Mendeley Desktop/Downloaded/Pavy et al. - 2012 - The heterogeneous levels of linkage disequilibrium in white spruce genes and comparative analysis with other conife.pdf:pdf},
issn = {0018067X},
journal = {Heredity},
keywords = {Picea,demography,linkage disequilibrium,nucleotide diversity,selection signatures,spruce},
month = {mar},
number = {3},
pages = {273--284},
pmid = {21897435},
publisher = {Nature Publishing Group},
title = {{The heterogeneous levels of linkage disequilibrium in white spruce genes and comparative analysis with other conifers}},
url = {www.nature.com/hdy},
volume = {108},
year = {2012}
}
@article{Mosca2014,
abstract = {Summary: Alpine ecosystems are facing rapid human-induced environmental changes, and so more knowledge about tree adaptive potential is needed. This study investigated the relative role of isolation by distance (IBD) versus isolation by adaptation (IBA) in explaining population genetic structure in Abies alba and Larix decidua, based on 231 and 233 single nucleotide polymorphisms (SNPs) sampled across 36 and 22 natural populations, respectively, in the Alps and Apennines. Genetic structure was investigated for both geographical and environmental groups, using analysis of molecular variance (AMOVA). For each species, nine environmental groups were defined using climate variables selected from a multiple factor analysis. Complementary methods were applied to identify outliers based on these groups, and to test for IBD versus IBA. AMOVA showed weak but significant genetic structure for both species, with higher values in L. decidua. Among the potential outliers detected, up to two loci were found for geographical groups and up to seven for environmental groups. A stronger effect of IBD than IBA was found in both species; nevertheless, once spatial effects had been removed, temperature and soil in A. alba, and precipitation in both species, were relevant factors explaining genetic structure. Based on our findings, in the Alpine region, genetic structure seems to be affected by both geographical isolation and environmental gradients, creating opportunities for local adaptation. {\textcopyright} 2013 New Phytologist Trust.},
author = {Mosca, Elena and Gonz{\'{a}}lez-Mart{\'{i}}nez, Santiago C. and Neale, David B.},
doi = {10.1111/nph.12476},
file = {:Users/s0784966/Library/Application Support/Mendeley Desktop/Downloaded/Mosca, Gonz{\'{a}}lez-Mart{\'{i}}nez, Neale - 2014 - Environmental versus geographical determinants of genetic structure in two subalpine conifers.pdf:pdf},
issn = {0028646X},
journal = {New Phytologist},
keywords = {Environmental gradient,Isolation by adaptation,Isolation by distance,Landscape genetics,Outlier locus detection},
month = {jan},
number = {1},
pages = {180--192},
publisher = {John Wiley {\&} Sons, Ltd},
title = {{Environmental versus geographical determinants of genetic structure in two subalpine conifers}},
url = {http://doi.wiley.com/10.1111/nph.12476},
volume = {201},
year = {2014}
}
@article{Mimura2007,
abstract = {Fossil pollen records suggest rapid migration of tree species in response to Quaternary climate warming. Long-distance dispersal and high gene flow would facilitate rapid migration, but would initially homogenize variation among populations. However, contemporary clinal variation in adaptive traits along environmental gradients shown in many tree species suggests that local adaptation can occur during rapid migration over just a few generations in interglacial periods. In this study, we compared growth performance and pollen genetic structure among populations to investigate how populations of Sitka spruce (Picea sitchensis) have responded to local selection along the historical migration route. The results suggest strong adaptive divergence among populations (average QST=0.61), corresponding to climatic gradients. The population genetic structure, determined by microsatellite markers (R ST=0.09; FST=0.11), was higher than previous estimates from less polymorphic genetic markers. The significant correlation between geographic and pollen haplotype genetic (RST) distances (r=0.73, P{\textless}0.01) indicates that the current genetic structure has been shaped by isolation-by-distance, and has developed in relatively few generations. This suggests relatively limited gene flow among populations on a recent timescale. Gene flow from neighboring populations may have provided genetic diversity to founder populations during rapid migration in the early stages of range expansion. Increased genetic diversity subsequently enhanced the efficiency of local selection, limiting gene flow primarily to among similar environments and facilitating the evolution of adaptive clinal variation along environmental gradients. {\textcopyright} 2007 Nature Publishing Group All rights reserved.},
author = {Mimura, M. and Aitken, S. N.},
doi = {10.1038/sj.hdy.6800987},
file = {:Users/s0784966/Library/Application Support/Mendeley Desktop/Downloaded/Mimura, Aitken - 2007 - Adaptive gradients and isolation-by-distance with postglacial migration in Picea sitchensis.pdf:pdf},
issn = {0018067X},
journal = {Heredity},
keywords = {Adaptation,Adaptive cline,Conifers,Gene flow,Population size,Range shift},
month = {aug},
number = {2},
pages = {224--232},
pmid = {17487214},
publisher = {Nature Publishing Group},
title = {{Adaptive gradients and isolation-by-distance with postglacial migration in Picea sitchensis}},
url = {www.nature.com/hdy},
volume = {99},
year = {2007}
}
@book{Walsh,
author = {Walsh, B. and Lynch, M.},
title = {{Evolution and Selection of Quantitative Traits}},
url = {https://global.oup.com/academic/product/evolution-and-selection-of-quantitative-traits-9780198830870?cc=ca{\&}lang=en{\&}}
}
@article{Wang2016,
abstract = {Large volumes of gridded climate data have become available in recent years including interpolated historical data from weather stations and future predictions from general circulation models. These datasets, however, are at various spatial resolutions that need to be converted to scales meaningful for applications such as climate change risk and impact assessments or sample-based ecological research. Extracting climate data for specific locations from large datasets is not a trivial task and typically requires advanced GIS and data management skills. In this study, we developed a software package, ClimateNA, that facilitates this task and provides a user-friendly interface suitable for resource managers and decision makers as well as scientists. The software locally downscales historical and future monthly climate data layers into scale-free point estimates of climate values for the entire North American continent. The software also calculates a large number of biologically relevant climate variables that are usually derived from daily weather data. ClimateNA covers 1) 104 years of historical data (1901-2014) in monthly, annual, decadal and 30-year time steps; 2) three paleoclimatic periods (Last Glacial Maximum, Mid Holocene and Last Millennium); 3) three future periods (2020s, 2050s and 2080s); and 4) annual time-series of model projections for 2011-2100. Multiple general circulation models (GCMs) were included for both paleo and future periods, and two representative concentration pathways (RCP4.5 and 8.5) were chosen for future climate data.},
author = {Wang, Tongli and Hamann, Andreas and Spittlehouse, Dave and Carroll, Carlos},
doi = {10.1371/journal.pone.0156720},
editor = {{\'{A}}lvarez, In{\'{e}}s},
file = {:Users/s0784966/Library/Application Support/Mendeley Desktop/Downloaded/Wang et al. - 2016 - Locally Downscaled and Spatially Customizable Climate Data for Historical and Future Periods for North America.pdf:pdf},
issn = {1932-6203},
journal = {PLOS ONE},
keywords = {Climate change,Meteorology,North America,Paleoclimatology,Paleoecology,Prisms,Software tools,Weather stations},
month = {jun},
number = {6},
pages = {e0156720},
publisher = {Public Library of Science},
title = {{Locally Downscaled and Spatially Customizable Climate Data for Historical and Future Periods for North America}},
url = {https://dx.plos.org/10.1371/journal.pone.0156720},
volume = {11},
year = {2016}
}
@article{WHITLOCK2005,
abstract = {The most commonly used method in evolutionary biology for combining information across multiple tests of the same null hypothesis is Fisher's combined probability test. This note shows that an alternative method called the weighted Z-test has more power and more precision than does Fisher's test. Furthermore, in contrast to some statements in the literature, the weighted Z-method is superior to the unweighted Z-transform approach. The results in this note show that, when combining P-values from multiple tests of the same hypothesis, the weighted Z-method should be preferred. {\textcopyright} 2005 European Society for Evolutionary Biology.},
author = {WHITLOCK, M. C.},
doi = {10.1111/j.1420-9101.2005.00917.x},
file = {:Users/s0784966/Library/Application Support/Mendeley Desktop/Downloaded/WHITLOCK - 2005 - Combining probability from independent tests the weighted Z-method is superior to Fisher's approach.pdf:pdf},
issn = {1010061X},
journal = {Journal of Evolutionary Biology},
keywords = {Combined probabilities,Fisher's procedure,Meta-analysis,Multiple comparisons,Stouffer's method,Weighted Z},
month = {aug},
number = {5},
pages = {1368--1373},
publisher = {John Wiley {\&} Sons, Ltd},
title = {{Combining probability from independent tests: the weighted Z-method is superior to Fisher's approach}},
url = {http://doi.wiley.com/10.1111/j.1420-9101.2005.00917.x},
volume = {18},
year = {2005}
}
@article{Hoban2016,
abstract = {Uncovering the genetic and evolutionary basis of local adaptation is a major focus of evolutionary biology. The recent development of cost-effective methods for obtaining high-quality genomescale data makes it possible to identify some of the loci responsible for adaptive differences among populations. Two basic approaches for identifying putatively locally adaptive loci have been developed and are broadly used: one that identifies loci with unusually high genetic differentiation among populations (differentiation outliermethods) and one that searches for correlations between local population allele frequencies and local environments (genetic-environment association methods). Here, we review the promises and challenges of these genome scanmethods, including correcting for the confounding influence of a species' demographic history, biases caused by missing aspects of the genome, matching scales of environmental data with population structure, and other statistical considerations. In each case, we make suggestions for best practices for maximizing the accuracy and efficiency of genome scans to detect the underlying genetic basis of local adaptation. With attention to their current limitations, genome scan methods can be an important tool in finding the genetic basis of adaptive evolutionary change.},
author = {Hoban, Sean and Kelley, Joanna L. and Lotterhos, Katie E. and Antolin, Michael F. and Bradburd, Gideon and Lowry, David B. and Poss, Mary L. and Reed, Laura K. and Storfer, Andrew and Whitlock, Michael C.},
doi = {10.1086/688018},
issn = {00030147},
journal = {American Naturalist},
keywords = {Differentiation outlier,FST outliers,Genetic-environment association,Genetics of adaptation,Genome scans,Local adaptation},
month = {oct},
number = {4},
pages = {379--397},
pmid = {27622873},
publisher = {University of Chicago Press},
title = {{Finding the genomic basis of local adaptation: Pitfalls, practical solutions, and future directions}},
url = {http://orcid.org/0000-0002-7731-605X;},
volume = {188},
year = {2016}
}
@article{Yeaman2016,
abstract = {When confronted with an adaptive challenge, such as extreme temperature, closely related species frequently evolve similar phenotypes using the same genes. Although such repeated evolution is thought to be less likely in highly polygenic traits and distantly related species, this has not been tested at the genome scale. We performed a population genomic study of convergent local adaptation among two distantly related species, lodgepole pine and interior spruce. We identified a suite of 47 genes, enriched for duplicated genes, with variants associated with spatial variation in temperature or cold hardiness in both species, providing evidence of convergent local adaptation despite 140 million years of separate evolution. These results show that adaptation to climate can be genetically constrained, with certain key genes playing nonredundant roles.},
author = {Yeaman, Sam and Hodgins, Kathryn A. and Lotterhos, Katie E. and Suren, Haktan and Nadeau, Simon and Degner, Jon C. and Nurkowski, Kristin A. and Smets, Pia and Wang, Tongli and Gray, Laura K. and Liepe, Katharina J. and Hamann, Andreas and Holliday, Jason A. and Whitlock, Michael C. and Rieseberg, Loren H. and Aitken, Sally N.},
doi = {10.1126/science.aaf7812},
file = {:Users/s0784966/Library/Application Support/Mendeley Desktop/Downloaded/Yeaman et al. - 2016 - Convergent local adaptation to climate in distantly related conifers.pdf:pdf},
issn = {10959203},
journal = {Science},
month = {sep},
number = {6306},
pages = {1431--1433},
pmid = {27708038},
publisher = {American Association for the Advancement of Science},
title = {{Convergent local adaptation to climate in distantly related conifers}},
url = {http://science.sciencemag.org/},
volume = {353},
year = {2016}
}
@article{Todesco2020,
abstract = {Species often include multiple ecotypes that are adapted to different environments1. However, it is unclear how ecotypes arise and how their distinctive combinations of adaptive alleles are maintained despite hybridization with non-adapted populations2–4. Here, by resequencing 1,506 wild sunflowers from 3 species (Helianthus annuus, Helianthus petiolaris and Helianthus argophyllus), we identify 37 large (1–100 Mbp in size), non-recombining haplotype blocks that are associated with numerous ecologically relevant traits, as well as soil and climate characteristics. Limited recombination in these haplotype blocks keeps adaptive alleles together, and these regions differentiate sunflower ecotypes. For example, haplotype blocks control a 77-day difference in flowering between ecotypes of the silverleaf sunflower H. argophyllus (probably through deletion of a homologue of FLOWERING LOCUS T (FT)), and are associated with seed size, flowering time and soil fertility in dune-adapted sunflowers. These haplotypes are highly divergent, frequently associated with structural variants and often appear to represent introgressions from other—possibly now-extinct—congeners. These results highlight a pervasive role of structural variation in ecotypic adaptation.},
author = {Todesco, Marco and Owens, Gregory L. and Bercovich, Natalia and L{\'{e}}gar{\'{e}}, Jean S{\'{e}}bastien and Soudi, Shaghayegh and Burge, Dylan O. and Huang, Kaichi and Ostevik, Katherine L. and Drummond, Emily B.M. and Imerovski, Ivana and Lande, Kathryn and Pascual-Robles, Mariana A. and Nanavati, Mihir and Jahani, Mojtaba and Cheung, Winnie and Staton, S. Evan and Mu{\~{n}}os, St{\'{e}}phane and Nielsen, Rasmus and Donovan, Lisa A. and Burke, John M. and Yeaman, Sam and Rieseberg, Loren H.},
doi = {10.1038/s41586-020-2467-6},
issn = {14764687},
journal = {Nature},
keywords = {Evolutionary genetics,Genome,Genomics,Plant genetics,Structural variation,wide association studies},
month = {aug},
number = {7822},
pages = {602--607},
pmid = {32641831},
publisher = {Nature Research},
title = {{Massive haplotypes underlie ecotypic differentiation in sunflowers}},
url = {https://doi.org/10.1038/s41586-020-2467-6},
volume = {584},
year = {2020}
}
@article{Morales2019,
abstract = {The study of parallel ecological divergence provides important clues to the operation of natural selection. Parallel divergence often occurs in heterogeneous environments with different kinds of environmental gradients in different locations, but the genomic basis underlying this process is unknown. We investigated the genomics of rapid parallel adaptation in the marine snail Littorina saxatilis in response to two independent environmental axes (crab-predation versus wave-action and low-shore versus high-shore). Using pooled whole-genome resequencing, we show that sharing of genomic regions of high differentiation between environments is generally low but increases at smaller spatial scales. We identify different shared genomic regions of divergence for each environmental axis and show that most of these regions overlap with candidate chromosomal inversions. Several inversion regions are divergent and polymorphic across many localities. We argue that chromosomal inversions could store shared variation that fuels rapid parallel adaptation to heterogeneous environments, possibly as balanced polymorphism shared by adaptive gene flow.},
author = {Morales, Hern{\'{a}}n E. and Faria, Rui and Johannesson, Kerstin and Larsson, Tomas and Panova, Marina and Westram, Anja M. and Butlin, Roger K.},
doi = {10.1126/sciadv.aav9963},
file = {:Users/s0784966/Library/Application Support/Mendeley Desktop/Downloaded/Morales et al. - 2019 - Genomic architecture of parallel ecological divergence Beyond a single environmental contrast.pdf:pdf},
issn = {23752548},
journal = {Science Advances},
month = {dec},
number = {12},
pages = {eaav9963},
pmid = {31840052},
publisher = {American Association for the Advancement of Science},
title = {{Genomic architecture of parallel ecological divergence: Beyond a single environmental contrast}},
url = {http://advances.sciencemag.org/},
volume = {5},
year = {2019}
}
@article{Noor2001,
abstract = {Recent genetic studies have suggested that many genes contribute to differences between closely related species that prevent gene exchange, particularly hybrid male sterility and female species preferences. We have examined the genetic basis of hybrid sterility and female species preferences in Drosophila pseudoobscura and Drosophila persimilis, two occasionally hybridizing North American species. Contrary to findings in other species groups, very few regions of the genome were associated with these characters, and these regions are associated also with fixed arrangement differences (inversions) between these species. From our results, we propose a preliminary genic model whereby inversions may contribute to the speciation process, thereby explaining the abundance of arrangement differences between closely related species that co-occur geographically. We suggest that inversions create linkage groups that cause sterility to persist between hybridizing taxa. The maintenance of this sterility allows the species to persist in the face of gene flow longer than without such inversions, and natural selection will have a greater opportunity to decrease the frequency of interspecies matings.},
author = {Noor, M. A.F. and Gratos, K. L. and Bertucci, L. A. and Reiland, J.},
doi = {10.1073/pnas.221274498},
file = {:Users/s0784966/Library/Application Support/Mendeley Desktop/Downloaded/Noor et al. - 2001 - Chromosomal inversions and the reproductive isolation of species.pdf:pdf},
issn = {00278424},
journal = {Proceedings of the National Academy of Sciences of the United States of America},
month = {oct},
number = {21},
pages = {12084--12088},
pmid = {11593019},
publisher = {National Academy of Sciences},
title = {{Chromosomal inversions and the reproductive isolation of species}},
url = {www.pnas.orgcgidoi10.1073pnas.221274498},
volume = {98},
year = {2001}
}
@misc{Rieseberg2001,
abstract = {Several authors have proposed that speciation frequently occurs when a population becomes fixed for one or more chromosomal rearrangements that reduce fitness when they are heterozygous. This hypothesis has little theoretical support because mutations that cause a large reduction in fitness can be fixed through drift only in small, inbred populations. Moreover, the effects of chromosomal rearrangements on fitness are unpredictable and vary significantly between plants and animals. I argue that rearrangements reduce gene flow more by suppressing recombination and extending the effects of linked isolation genes than by reducing fitness. This unorthodox perspective has significant implications for speciation models and for the outcomes of contact between neospecies and their progenitor(s).},
author = {Rieseberg, Loren H.},
booktitle = {Trends in Ecology and Evolution},
doi = {10.1016/S0169-5347(01)02187-5},
issn = {01695347},
keywords = {Biology,Endocrinology,Genetics,chromosomal rearrangements,gene flow,karyotyp,meiotic drive,neospecies,reinforcement,speciation},
month = {jul},
number = {7},
pages = {351--358},
pmid = {11403867},
publisher = {Elsevier},
title = {{Chromosomal rearrangements and speciation}},
url = {http://www.cell.com/article/S0169534701021875/fulltext http://www.cell.com/article/S0169534701021875/abstract https://www.cell.com/trends/ecology-evolution/abstract/S0169-5347(01)02187-5},
volume = {16},
year = {2001}
}
@article{Gautier2015,
abstract = {In population genomics studies, accounting for the neutral covariance structure across population allele frequencies is critical to improve the robustness of genome-wide scan approaches. Elaborating on the BayEnv model, this study investigates several modeling extensions (i) to improve the estimation accuracy of the population covariance matrix and all the related measures, (ii) to identify significantly overly differentiated SNPs based on a calibration procedure of the XtX statistics, and (iii) to consider alternative covariate models for analyses of association with population-specific covariables. In particular, the auxiliary variable model allows one to deal with multiple testing issues and, providing the relative marker positions are available, to capture some linkage disequilibrium information. A comprehensive simulation study was carried out to evaluate the performances of these different models. Also, when compared in terms of power, robustness, and computational efficiency to five other state-of-the-art genome-scan methods (BayEnv2, BayScEnv, BayScan, FLK, and LFMM), the proposed approaches proved highly effective. For illustration purposes, genotyping data on 18 French cattle breeds were analyzed, leading to the identification of 13 strong signatures of selection. Among these, four (surrounding the KITLG, KIT, EDN3, and ALB genes) contained SNPs strongly associated with the piebald coloration pattern while a fifth (surrounding PLAG1) could be associated to morphological differences across the populations. Finally, analysis of Pool-Seq data from 12 populations of Littorina saxatilis living in two different ecotypes illustrates how the proposed framework might help in addressing relevant ecological issues in nonmodel species. Overall, the proposed methods define a robust Bayesian framework to characterize adaptive genetic differentiation across populations. The BayPass program implementing the different models is available at http:// www1.montpellier.inra.fr/CBGP/software/baypass/.},
author = {Gautier, Mathieu},
doi = {10.1534/genetics.115.181453},
file = {:Users/s0784966/Library/Application Support/Mendeley Desktop/Downloaded/Gautier - 2015 - Genome-wide scan for adaptive divergence and association with population-specific covariates.pdf:pdf},
issn = {19432631},
journal = {Genetics},
keywords = {Association studies,Bayesian statistics,Genome scan,Linkage disequilibrium,Pool-Seq},
month = {dec},
number = {4},
pages = {1555--1579},
pmid = {26482796},
publisher = {Genetics Society of America},
title = {{Genome-wide scan for adaptive divergence and association with population-specific covariates}},
url = {www.genetics.org/lookup/suppl/},
volume = {201},
year = {2015}
}
@article{Coop2010,
abstract = {Loci involved in local adaptation can potentially be identified by an unusual correlation between allele frequencies and important ecological variables or by extreme allele frequency differences between geographic regions. However, such comparisons are complicated by differences in sample sizes and the neutral correlation of allele frequencies across populations due to shared history and gene flow. To overcome these difficulties, we have developed a Bayesian method that estimates the empirical pattern of covariance in allele frequencies between populations from a set of markers and then uses this as a null model for a test at individual SNPs. In our model the sample frequencies of an allele across populations are drawn from a set of underlying population frequencies; a transform of these population frequencies is assumed to follow a multivariate normal distribution. We first estimate the covariance matrix of this multivariate normal across loci using a Monte Carlo Markov chain. At each SNP, we then provide a measure of the support, a Bayes factor, for a model where an environmental variable has a linear effect on the transformed allele frequencies compared to a model given by the covariance matrix alone. This test is shown through power simulations to outperform existing correlation tests. We also demonstrate that our method can be used to identify SNPs with unusually large allele frequency differentiation and offers a powerful alternative to tests based on pairwise or global FST. Software is available at http://www.eve.ucdavis.edu/gmcoop/. Copyright {\textcopyright} 2010 by the Genetics Society of America.},
author = {Coop, Graham and Witonsky, David and {Di Rienzo}, Anna and Pritchard, Jonathan K.},
doi = {10.1534/genetics.110.114819},
file = {:Users/s0784966/Library/Application Support/Mendeley Desktop/Downloaded/Coop et al. - 2010 - Using environmental correlations to identify loci underlying local adaptation(3).pdf:pdf},
issn = {00166731},
journal = {Genetics},
month = {aug},
number = {4},
pages = {1411--1423},
pmid = {20516501},
publisher = {Genetics},
title = {{Using environmental correlations to identify loci underlying local adaptation}},
url = {http://www.},
volume = {185},
year = {2010}
}
@article{Berg2019,
abstract = {Several recent papers have reported strong signals of selection on European polygenic height scores. These analyses used height effect estimates from the GIANT consortium and replication studies. Here, we describe a new analysis based on the the UK Biobank (UKB), a large, independent dataset. We find that the signals of selection using UKB effect estimates are strongly attenuated or absent. We also provide evidence that previous analyses were confounded by population stratification. Therefore, the conclusion of strong polygenic adaptation now lacks support. Moreover, these discrepancies highlight (1) that methods for correcting for population stratification in GWAS may not always be sufficient for polygenic trait analyses, and (2) that claims of differences in polygenic scores between populations should be treated with caution until these issues are better understood. Editorial note: This article has been through an editorial process in which the authors decide how to respond to the issues raised during peer review. The Reviewing Editor's assessment is that all the issues have been addressed (see decision letter).},
author = {Berg, Jeremy J. and Harpak, Arbel and Sinnott-Armstrong, Nasa and Joergensen, Anja Moltke and Mostafavi, Hakhamanesh and Field, Yair and Boyle, Evan August and Zhang, Xinjun and Racimo, Fernando and Pritchard, Jonathan K. and Coop, Graham},
doi = {10.7554/eLife.39725},
issn = {2050084X},
journal = {eLife},
month = {mar},
pmid = {30895923},
publisher = {eLife Sciences Publications Ltd},
title = {{Reduced signal for polygenic adaptation of height in UK biobank}},
volume = {8},
year = {2019}
}
@article{Sohail2019,
abstract = {Genetic predictions of height differ among human populations and these differences have been interpreted as evidence of polygenic adaptation. These differences were first detected using SNPs genome-wide significantly associated with height, and shown to grow stronger when large numbers of sub-significant SNPs were included, leading to excitement about the prospect of analyzing large fractions of the genome to detect polygenic adaptation for multiple traits. Previous studies of height have been based on SNP effect size measurements in the GIANT Consortium meta-analysis. Here we repeat the analyses in the UK Biobank, a much more homogeneously designed study. We show that polygenic adaptation signals based on large numbers of SNPs below genome-wide significance are extremely sensitive to biases due to uncorrected population stratification. More generally, our results imply that typical constructions of polygenic scores are sensitive to population stratification and that population-level differences should be interpreted with caution. Editorial note: This article has been through an editorial process in which the authors decide how to respond to the issues raised during peer review. The Reviewing Editor's assessment is that all the issues have been addressed (see decision letter).},
author = {Sohail, Mashaal and Maier, Robert M. and Ganna, Andrea and Bloemendal, Alex and Martin, Alicia R. and Turchin, Michael C. and Chiang, Charleston W.K. and Hirschhorn, Joel and Daly, Mark J. and Patterson, Nick and Neale, Benjamin and Mathieson, Iain and Reich, David and Sunyaev, Shamil R.},
doi = {10.7554/eLife.39702},
file = {:Users/s0784966/Library/Application Support/Mendeley Desktop/Downloaded/Sohail et al. - 2019 - Polygenic adaptation on height is overestimated due to uncorrected stratification in genome-wide association stud.pdf:pdf},
issn = {2050084X},
journal = {eLife},
month = {mar},
pmid = {30895926},
publisher = {eLife Sciences Publications Ltd},
title = {{Polygenic adaptation on height is overestimated due to uncorrected stratification in genome-wide association studies}},
volume = {8},
year = {2019}
}
@article{Booker2020,
abstract = {Genome scans can potentially identify genetic loci involved in evolutionary processes such as local adaptation and gene flow. Here, we show that recombination rate variation across a neutrally evolving genome gives rise to mixed sampling distributions of mean FST ((Formula presented.)), a common population genetic summary statistic. In particular, we show that in regions of low recombination the distribution of (Formula presented.) estimates has more variance and a longer tail than in more highly recombining regions. Determining outliers from the genome-wide distribution without taking local recombination rate into consideration may therefore increase the frequency of false positives in low recombination regions and be overly conservative in more highly recombining ones. We perform genome scans on simulated and empirical Drosophila melanogaster data sets and, in both cases, find patterns consistent with this neutral model. Similar patterns are observed for other summary statistics used to capture variation in the coalescent process. Linked selection, particularly background selection, is often invoked to explain heterogeneity in (Formula presented.) across the genome, but here we point out that even under neutrality, statistical artefacts can arise due to variation in recombination rate. Our results highlight a flaw in the design of genome-scan studies and suggest that without estimates of local recombination rate, interpreting the genomic landscape of any summary statistic that captures variation in the coalescent process will be very difficult.},
author = {Booker, Tom R. and Yeaman, Sam and Whitlock, Michael C.},
doi = {10.1111/mec.15501},
issn = {0962-1083},
journal = {Molecular Ecology},
keywords = {adaptation,molecular evolution,population genetics—empirical,population genetics—theoretical},
month = {nov},
number = {22},
pages = {4274--4279},
publisher = {Blackwell Publishing Ltd},
title = {{Variation in recombination rate affects detection of outliers in genome scans under neutrality}},
url = {https://onlinelibrary.wiley.com/doi/10.1111/mec.15501},
volume = {29},
year = {2020}
}
@article{Lotterhos2019-ki,
abstract = {Recently, there has been an increasing interest in identifying
the role that regions of low recombination or inversion play in
adaptation of species to local environments. Many examples of
groups of adapted genes located within inversions are arising in
the literature, in part inspired by theory that predicts the
evolution of these so-called ``supergenes.'' We still, however,
have a poor understanding of how genomic heterogeneity, such as
varying rates of recombination, may confound signals of
selection. Here, I evaluate the effect of neutral inversions and
recombination variation on genome scans for selection, including
tests for selective sweeps, differentiation outlier tests, and
association tests. There is considerable variation among methods
in their performance, with some methods being unaffected and some
showing elevated false positive signals within a neutral
inversion or region of low recombination. In some cases the false
positive signal can be dampened or removed, if it is possible to
use a quasi-independent set of SNPs to parameterize the model
before performing the test. These results will be helpful to
those seeking to understand the importance of regions of low
recombination in adaptation.},
author = {Lotterhos, Katie E},
journal = {G3},
keywords = {adaptation; linkage disequilibrium; population str},
month = {jun},
number = {6},
pages = {1851--1867},
title = {{The Effect of Neutral Recombination Variation on Genome Scans for Selection}},
volume = {9},
year = {2019}
}
@article{Barton1986,
abstract = {Suppose that selection acts at one or more loci to maintain genetic differences between hybridising populations. Then, the flow of alleles at a neutral marker locus which is linked to these selected loci will be impeded. We define and calculate measures of the barrier to gene flow between two distinct demes, and across a continuous habitat. In both cases, we find that in order for gene flow to be significantly reduced over much of the genome, hybrids must be substantially less fit, and the number of genes involved in building the barrier must be so large that the majority of other genes become closely linked to some locus which is under selection. This conclusion is not greatly affected by the pattern of epistasis, or the position of the marker locus along the chromosome.},
author = {Barton, Nick and Bengtsson, Bengt Olle},
doi = {10.1038/hdy.1986.135},
file = {:Users/s0784966/Library/Application Support/Mendeley Desktop/Downloaded/Barton, Bengtsson - 1986 - The barrier to genetic exchange between hybridising populations(2).pdf:pdf},
issn = {13652540},
journal = {Heredity},
number = {3},
pages = {357--376},
title = {{The barrier to genetic exchange between hybridising populations}},
volume = {57},
year = {1986}
}
@article{Haller2019-jv,
abstract = {With the desire to model population genetic processes under
increasingly realistic scenarios, forward genetic simulations
have become a critical part of the toolbox of modern
evolutionary biology. The SLiM forward genetic simulation
framework is one of the most powerful and widely used tools in
this area. However, its foundation in the Wright--Fisher model
has been found to pose an obstacle to implementing many types of
models; it is difficult to adapt the Wright--Fisher model, with
its many assumptions, to modeling ecologically realistic
scenarios such as explicit space, overlapping generations,
individual variation in reproduction, density-dependent
population regulation, individual variation in dispersal or
migration, local extinction and recolonization, mating between
subpopulations, age structure, fitness-based survival and hard
selection, emergent sex ratios, and so forth. In response to
this need, we here introduce SLiM 3, which contains two key
advancements aimed at abolishing these limitations. First, the
new non-Wright--Fisher or ``nonWF'' model type provides a much
more flexible foundation that allows the easy implementation of
all of the above scenarios and many more. Second, SLiM 3 adds
support for continuous space, including spatial interactions and
spatial maps of environmental variables. We provide a conceptual
overview of these new features, and present several example
models to illustrate their use.},
author = {Haller, Benjamin C and Messer, Philipp W},
journal = {Mol. Biol. Evol.},
month = {mar},
number = {3},
pages = {632--637},
publisher = {Narnia},
title = {{{\{}SLiM{\}} 3: Forward Genetic Simulations Beyond the {\{}Wright--Fisher{\}} Model}},
volume = {36},
year = {2019}
}
@article{Sakamoto2019,
abstract = {Divergent selection works when an allele establishes in the subpopulations in which it is adaptive, but not in the ones in which it is deleterious. While such a locally adaptive allele is maintained, the target locus of selection works as a genetic barrier to gene flow or a barrier locus. The genetic divergence (or FST) around the barrier locus can be maintained, while in other regions of the genome, genetic variation can be mixed by gene flow or migration. In this work, we consider theoretically the evolutionary process of a barrier locus, from its birth to stable preservation. Under a simple two-population model, we use a diffusion approach to obtain analytical expressions for the probability of initial establishment of a locally adaptive allele, the reduction of genetic variation due to the spread of the adaptive allele, and the process to the development of a sharp peak of divergence (genomic island of divergence). Our results will be useful to understanding how genomes evolve through local adaptation and divergent selection.},
author = {Sakamoto, Takahiro and Innan, Hideki},
doi = {10.1534/genetics.119.302311},
file = {:Users/s0784966/Library/Application Support/Mendeley Desktop/Downloaded/Sakamoto, Innan - 2019 - The evolutionary dynamics of a genetic barrier to gene flow From the establishment to the emergence of a pea(3).pdf:pdf},
issn = {19432631},
journal = {Genetics},
keywords = {Diffusion theory,Divergent selection,Gene flow,Migration,Population genetics,Speciation},
month = {jun},
number = {4},
pages = {1383--1398},
publisher = {Genetics Society of America},
title = {{The evolutionary dynamics of a genetic barrier to gene flow: From the establishment to the emergence of a peak of divergence}},
volume = {212},
year = {2019}
}
@book{RN173,
address = {Greenwood Village, Colorado},
author = {Charlesworth, B and Charlesworth, D},
pages = {734},
publisher = {Roberts {\&} Company},
title = {{Elements of Evolutionary Genetics}},
type = {Book},
year = {2010}
}
@article{Haller2019-za,
author = {Haller, Benjamin C and Galloway, Jared and Kelleher, Jerome and Messer, Philipp W and Ralph, Peter L},
journal = {Mol. Ecol. Resour.},
number = {2},
pages = {552--566},
publisher = {Wiley Online Library},
title = {{Tree-sequence recording in {\{}SLiM{\}} opens new horizons for forward-time simulation of whole genomes}},
volume = {19},
year = {2019}
}
@article{Kelleher2016-zz,
abstract = {A central challenge in the analysis of genetic variation is to
provide realistic genome simulation across millions of samples.
Present day coalescent simulations do not scale well, or use
approximations that fail to capture important long-range linkage
properties. Analysing the results of simulations also presents a
substantial challenge, as current methods to store genealogies
consume a great deal of space, are slow to parse and do not take
advantage of shared structure in correlated trees. We solve these
problems by introducing sparse trees and coalescence records as
the key units of genealogical analysis. Using these tools, exact
simulation of the coalescent with recombination for
chromosome-sized regions over hundreds of thousands of samples is
possible, and substantially faster than present-day approximate
methods. We can also analyse the results orders of magnitude more
quickly than with existing methods.},
author = {Kelleher, Jerome and Etheridge, Alison M and McVean, Gilean},
journal = {PLoS Comput. Biol.},
month = {may},
number = {5},
pages = {e1004842},
title = {{Efficient Coalescent Simulation and Genealogical Analysis for Large Sample Sizes}},
volume = {12},
year = {2016}
}
@article{Weir1984-tc,
author = {Weir, Bruce S and Cockerham, C Clark},
journal = {Evolution},
number = {6},
pages = {1358--1370},
publisher = {Wiley Online Library},
title = {{Estimating F-statistics for the analysis of population structure}},
volume = {38},
year = {1984}
}
